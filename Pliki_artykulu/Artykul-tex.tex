\documentclass[12pt,a4paper,final,oneside,onecolumn,titlepage]{article}
\usepackage{times}
\usepackage{geometry}
\usepackage{fancyhdr}
\usepackage{setspace}
\usepackage{natbib}
\usepackage{amssymb}
\usepackage{graphicx}
\usepackage{sectsty}
\usepackage[table,xcdraw]{xcolor}
\usepackage{polski}
\usepackage[utf8]{inputenc}
\usepackage[T1]{fontenc}
\usepackage[pagewise]{lineno}
\newgeometry{tmargin=2.5cm, bmargin=2.5cm, lmargin=2.5cm, rmargin=2.5cm}
\setlength{\parindent}{3in}
\setlength{\parskip}{0pt}
\linenumbers
\doublespacing
\sectionfont{\centering}
\renewcommand{\bibsection}{\section*{\large{\textbf{\textsc{\centering{Literatura}}}}}}

\begin{document}
\pagestyle{fancy}
\fancyhead{}
\fancyfoot{}
\chead{Rola platformy Instagram w odbiorze własnego wyglądu}
\rhead{\thepage}
\bibliographystyle{apalike}
\begin{titlepage}
  \thispagestyle{empty}
  \rhead{\thepage}
  \begin{center}
  \vspace*{1cm}
  \Large
  \textbf{\textsc{Rola platformy Instagram w odbiorze własnego wyglądu:\\ Badanie zależności między modyfikowaniem swojego wizerunku, a oceną postrzegania swojego ciała wśród kobiet\\}}
  \vspace{1.5cm}
  \textit{Sara Pasturczak, Aleksandra Popowska, Wiktor Warchałowski\\}
  Wydział Nauk o Zdrowiu, Gdański Uniwersytet Medyczny\\
  \vspace{3cm}
  Praca zaliczeniowa z przedmiotu \\ Metodologia Badań Psychologicznych \\ napisana pod kierunkiem dr Małgorzaty Basińskiej\\
  \vspace{3cm}
  Gdańsk, 24 Maja 2022
  \end{center}
\end{titlepage}
\begin{center}
  \vspace*{0.5cm}
  \large{\textbf{\textsc{Abstrakt}}}
\end{center}
\paragraph{}
Celem niniejszego artykułu jest zbadanie problematyki związanej z zależnością mediów społecznościowych i samoakceptacji. Artykuł ten sprawdza czy fakt modyfikacji zdjęć własnego wizerunku na platformie Instagram ma związek z poziomem zadowolenia z własnego ciała u młodych kobiet. Badanie zostało przeprowadzone na 76 użytkowniczkach platformy Instagram, najpopularniejszego serwisu społecznościowego wśród osób młodych. Postrzeganie własnego ciała zostało opisane jako wynik polskiej adaptacji 10 itemowego kwestionariusza Body Appreciation Scale 2, zaś fakt modyfikacji lub jego brak był podstawą do przydzielenia badanych do dwóch grup, które zostały ze sobą porównane. Średnie oraz mediany porównanych grup nie różniły się od siebie, zaś test t studenta nie wykazał statystycznie istotnej różnicy pomiędzy grupami. Współczynnik d Cohena pokazał brak efektu. Z tego powodu przyjęta została hipoteza alternatywna mówiąca o braku zależności.\\
\textit{Słowa kluczowe: media społecznościowe, postrzeganie siebie, ocena ciała, modyfikacja zdjęć}
\newpage
\begin{center}
\section*{\large{\textbf{\textsc{Wstęp}}}}
\end{center}
\paragraph{}
W dzisiejszych czasach coraz więcej osób ma dostęp do internetu, a co za tym idzie - również do portali społecznościowych, które cieszą się ogromnym zainteresowaniem. Najpopularniejszym portalem wśród amerykańskich nastolatków jest Instagram. Pozwala on na kontakty z rówieśnikami, wymianę informacji, a także wyrażanie siebie \citep{longobardi_follow_2020}. Jednakże tak powszechne użycie mediów społecznościowych prowokuje u młodych osób zachowania koncentrujące się na dążeniu do statusu w sieci (ang. \textit{digital status seeking}), czyli dążenia do popularności w Internecie mierzonej ilością polubień, komentarzy, udostępnień i obserwatorów \citep{nesi_search_2019}. Takie działania często prowadzą do zachowań ryzykownych, a samo wystawienie się na tego typu sieciową rywalizację, pozwala badaczom wysunąć wniosek, że Instagram jest najbardziej szodliwym serwisem społecznościowym. Jego nadmierne używanie może prowadzić nawet do problemów psychicznych czy zaburzeń postrzegania swojego ciała \citep{royal_society_for_public_health_status_2017}. Oprócz samego udziału młodych osób w kształtowaniu treści mediów społecznościowych, na tworzenie ich zawartości mają również wpływ czynniki socjokulturowe \citep{giorgianni_consumer_2020}. Kultura zachodnia od wieków ma tendencję do uprzedmiatawiania kobiet, a także cały czas zwraca uwagę na ważność wyglądu zewnętrznego \citep{lyu_travel_2016}. Ze względu na tak utarte kulturowe stereotypy, większość środków masowego przekazu powiela je i tworzy wizerunek ideału szczupłego ciała. Promowanie idelanych modelek i nagradzanie tego typu wizerunku w mediach jest czymś powszechnym. Stanowi to często główny powód niezadowolenia młodych kobiet ze swojego ciała, co z kolei może prowadzić do zaburzeń odżywiania \citep{grabe_role_2008}. Problemy te wynikają z dążenia do perfekcyjnego wyglądu, w celu sprostania oczekiwaniom społeczeństwa. Aby stworzyć złudzenie idealnego ciała w mediach społecznościowych, kobiety modyfikują swoje zdjęcia i często stanowi to dla nich mechanizm radzenia sobie z brakiem akceptacji siebie \citep{lyu_travel_2016}. Duży wpływ na poczucie zadowolenia z własnego wyglądu ma również porównywanie się z innymi osobami. Zetknięcie się przez młode dziewczęta ze zmodyfikowanymi zdjęciami rówieśników prowadzi do sytuacji, w których postrzegają one siebie bardziej negatywnie. Natomiast zdjęcia, które nie są przez rówieśników w żaden sposób zmienione, nie są źródłem takich skłonności \citep{kleemans_picture_2018}. Pomimo wielu przeprowadzonych badań potwierdzających szkodliwość platformy Instagram istnieją i takie, które zaprzeczają jego krzywdzącemu wpływowi. \citet{mclean_selfies_2015} w swoim badaniu pokazali, że publikowanie swoich zdjęć może wpływać na młode kobiety dwojako - mogą one zawyżać ocenę na temat swojego ciała, lub przyczyniać się do niezadowolenia z własnego wyglądu. Oprócz tego wyniki najnowszych badań pokazują, że coraz więcej osób zaczyna dostrzegać piękno sylwetek i twarzy, które wyraźnie nie wpisują się kanon piękna od lat funkcjonujący w kulturze zachodniej. Rozpowszechniający się ruch \textit{body positivity} skłania użytkowników mediów społecznościowych do pokazywania się takim, jakim się jest, do porzucenia poczucia wstydu oraz do kochania swojego ciała. Koncept ten sprzyja bardziej pozytywnemu postrzeganiu samego siebie, ponieważ jego zwolennicy uświadamiają osobom zmagającym się z niską samooceną, że każdy posiada niedoskonałości, które mogą być piękne i wyjątkowe \citep{LAZUKA202085}. Wyniki te mogą zaprzeczać jakiejkolwiek zależności między oceną własnego ciała a korzystaniem z mediów społecznościowych. Z tego powodu, celem niniejszego artykułu jest sprawdzenie czy fakt modyfikacji zdjęć własnego wizerunku na platformie Instagram ma związek z poziomem zadowolenia z własnego ciała u młodych kobiet. Przyjęta została hipoteza, iż fakt modyfikacji zdjęć będzie występował częściej u osób posiadających niskie zadowolenie z własnego wyglądu. Wynika to z faktu, że modyfikacja zdjęć może być mechanizmem radzenia sobie przez niektórych z negatywną samooceną \citep{kleemans_picture_2018,lyu_travel_2016}. Oprócz tego na tworzenie fałszywego wizerunku w mediach społecznościowych wpływa internalizacja ideału szczupłej sylwetki, która koreluje z negatywną oceną własnego ciała \citep{blowers_relationship_2003}.
\begin{center}
\section*{\large{\textbf{\textsc{Metoda}}}}
\end{center}
\subsection*{\normalsize{\textbf{Osoby badane}}}
\paragraph{}
Uczestnicy zostali wybrani metodą doboru przypadkowego. Osoby te otrzymały link do internetowego kwestionariusza i dobrowolnie go wypełniły. Łącznie kwestionariusz wypełniło 91 osób w wieku pomiędzy 18 a 25 rokiem życia. Jednakże, 5 osób zaznaczyło płeć inną niż żeńska, 2 inne osoby zaś przyznały, że nie mają konta na platformie Instagram. Aby wypełnić założenia badania, osoby te (łącznie 7) zostały odrzucone z próby. W założeniach przewidziane było badanie osób łącznie z próbą kontrolną (osób, które nie publikują zdjęć swojego wizerunku) jednakże, ze względu na małą ilość takich osób (8) zostały one odrzucone - porównywalność i reprezentatywność grup nie zostałaby zachowana. Ostatecznie próba badana wyniosła 76 uczestników. Osoby badane w znacznej części pochodziły z miast o populacji powyżej 100 tys. mieszkańców. Były to również w większości jednostki pobierające naukę.
\subsection*{\normalsize{\textbf{Etyka}}}
\paragraph{}
Zgoda etyczna na przeprowadzenie badania została otrzymana od prowadzącego przedmiot Metodologia Badań Psychologicznych, realizowanym na Gdańskim Uniwersytecie Medycznym. Zarówno wykorzystany przez nas kwestionariusz, jak i sama metoda badania nie budzą żadnych etycznych wątpliwości. Kwestionariusz nie zbiera danych wrażliwych, a jego autorzy wyrazili pełną zgodę na użycie go przez nas w ramach zajęć prowadzonych na naszej uczelni. Uczestnicy zostali poinformowani o celu badania oraz o jego naturze. Zapewniono ich również, iż udział w badaniu jest zupełnie dobrowolny i w każdej chwili mogą z niego zrezygnować, przez cały czas pozostając anonimowym. Pozyskane zostało też potwierdzenie, że wszyscy uczestnicy ukończyli 18 lat i mają prawo do wyrażenia samodzielnej, świadomej zgody na udział w badaniu. 
\subsection*{\normalsize{\textbf{Zastosowane narzędzia i operacjonalizacja}}}
\paragraph{}
Zmienna niezależna, jaką jest modyfikowanie lub niemodyfikowanie zdjęć publikowanych na platformie Instagram została zoperacjonalizowana jako wszelkie zabiegi dokonywane w celu polepszenia swojego wyglądu na zdjęciu. Było to opisane jako twierdząca lub przecząca odpowiedź na pytanie dotyczące modyfikacji zdjęć w brzmieniu ,,Czy kiedykolwiek, w jakikolwiek sposób zmodyfikowałaś zdjęcie swojego wizerunku na platformie Instagram tak, aby prezentować się na nim lepiej (prosimy o nieuwzględnianie filtrów, które jedynie zmieniają kolory na zdjęciach)?''. Zmienna zależna, jaką jest postrzeganie swojego ciała, została zoperacjonalizowana jako wynik kwestionariusza Body Appreciation Scale 2 (BAS-2) T. Tylka i N. Wood-Barcalow w polskiej adaptacji M. Razmus i W. Razmus. Wyższy wynik sumaryczny uzyskany przez odpowiedzi na pytania w pięciostopniowej skali Likerta jest interpretowany jako lepsze postrzeganie swojego ciała i posiadanie wyższej samooceny \citep{razmus_evaluating_2017,tylka_body_2015}.
\subsection*{\normalsize{\textbf{Procedura badania}}}
\paragraph{}
Badanie kwestionariuszowe zostało przeprowadzone przez Internet za pomocą Kwestionariuszy Google. Każdy badany musiał udzielić zgody na badanie oraz odpowiedzieć na 6 pytań dotyczących danych demograficznych, 1 pytanie dotyczące modyfikowania zdjęć na platformie Instagram oraz wypełnić kwestionariusz Body Appreciation Scale 2 (BAS-2) T. Tylka i N. Wood-Barcalow w polskiej adaptacji M. Razmus i W. Razmus składający się z 10 itemów. Dane uzyskane w kwestionariuszu zostały zsumowane i na tej podstawie zostały porównane dwie grupy badanych (osoby modyfikujące oraz niemodyfikujące swoich zdjęć umieszczanych na platfromie Instagram).
\subsection*{\normalsize{\textbf{Analiza statystyczna}}}
\paragraph{}
W celu udzielenia odpowiedzi na postawione pytanie badawcze oraz przetestowania postawionej hipotezy przeprowadzono analizy statystyczne przy użyciu języka programowania i środowiska obliczeniowego R Project for Statistical Computing \citep{Rstudio}. Pierwszym wykonanym zabiegiem statystycznym było sprawdzenie normalności rozkładu testem Shapiro-Wilka, aby być w stanie dobrać odpowiednio następne testy statystyczne. Za poziom istotności przyjęto $\alpha$ = 0.05. Wartość \textit{p} testu Shapiro-Wilka wynosi \textit{p = 0.01891}, co wskazuje na brak normalnej dystrybucji danych: \textit{p <} $\alpha$. Do zwizualizowania rozkładu wyników stworzono również histogram (\textit{Rysunek 1}). Z tego powodu testem statystycznym, który został wybrany dla porównania badanych grup został test t studenta dla grup niezależnych z poprawką Welcha \citep{welch_generalization_1947}. Oprócz tego policzone zotały podstawowe statystyki opisowe oraz siła efektu za pomocą współczynnika d Cohena.
\begin{figure}[h!]
\caption{\textit{Histogram przedstawiający rozkład wyników w obu badanych grupach}}
\centering
\includegraphics[scale=0.25]{histogram_all}
\end{figure}
\pagebreak
\section*{\large{\textbf{\textsc{Wyniki}}}}
\paragraph{}
Po sprawdzeniu normalności rozkładu i na tej podstawie wyboru testu statystycznego obliczone zostały podstawowe statystyki opisowe dla obu grup. Były to miary tendencji centralnej - średnia oraz mediana, jak również miary zróżnicowania - odchylenie standardowe oraz rozstęp. Wyniki podstawowych działań statystycznych zostały przedstawione w \textit{tabeli 1}. \\
\begin{table}[h!]
\caption{\textit{Wyniki podstawowych parametrów statystyki opisowej}}
\begin{tabular}{lccccccc}
\textbf{Modyfikacja} & \textbf{N}                                        & \textbf{Średnia}                                    & \textbf{Mediana}                                    & \textbf{\begin{tabular}[c]{@{}c@{}}Odchylenie\\ Standardowe\end{tabular}} & \textbf{Max}                                      & \textbf{Min}                                      & \textbf{Rozstęp}                                  \\
Tak                  & {\color[HTML]{333333} 25}                         & {\color[HTML]{333333} 32.7}                         & {\color[HTML]{333333} 32}                          & {\color[HTML]{333333} 9.6}                                                 & {\color[HTML]{333333} 49}                         & {\color[HTML]{333333} 11}                         & {\color[HTML]{333333} 38}                         \\
Nie                  & \cellcolor[HTML]{F2F2F2}{\color[HTML]{333333} 51} & \cellcolor[HTML]{F2F2F2}{\color[HTML]{333333} 32.7} & \cellcolor[HTML]{F2F2F2}{\color[HTML]{333333} 34} & \cellcolor[HTML]{F2F2F2}{\color[HTML]{333333} 9.59}                         & \cellcolor[HTML]{F2F2F2}{\color[HTML]{333333} 50} & \cellcolor[HTML]{F2F2F2}{\color[HTML]{333333} 11} & \cellcolor[HTML]{F2F2F2}{\color[HTML]{333333} 39}
\end{tabular}
\end{table}
\paragraph{}
Poza podstawowymi miarami rozkładu wyników, policzone zostały testy statystyczne wskazujące na ewentualne zależności pomiędzy porównanymi grupami. Wartość \textit {p} testu t studenta z poprawką Welcha wyniosła \textit{p = 0.9912}. Następnie sprawdzony został ,,stopień do jakiego badane zjawisko istnieje’’ \citep[s. 5]{cohen_statistical_1977} za pomocą współczynnika d Cohena, którego wartość wyniosła 0.002698416. Aby wizualnie przedstawić porównanie badanych grup, na podstawie obliczonych parametrów został stworzony wykres pudełkowy, który przedstawiony jest na \textit{Rysunku 2}.
\begin{figure}[h!]
\caption{\textit{Wykres pudełkowy porównujący badane grupy}}
\centering
\includegraphics[scale=0.25]{boxwhisk}
\end{figure}
\paragraph{}
Hipoteza postawiona w całym badaniu została przyjęta jako \textit{Hipoteza zerowa (H$_0$)} w analizie statystycznej testującej hipotezy, zaś \textit{Hipoteza alternatywna (H$_1$)} jest zaprzeczeniem \textit{H$_0$}. Po porównaniu wartości \textit{p} testu t studenta z przyjętym poziomem istotności można pokazać, że \textit{H$_0$} musi zostać odrzucona, zaś \textit{Hipoteza alternatywna} została przyjęta. Siła zaobserwowanego efektu d Cohena jest mniejsza od 0.2 i jest bliska zera co wskazuje na brak występowania efektu pomiędzy grupami.
\section*{\large{\textbf{\textsc{Dyskusja}}}}
\paragraph{}
Badanie miało na celu sprawdzenie zależności pomiędzy postrzeganiem samego siebie, a modyfikowaniem zdjęć swojego wizerunku na platformie Instagram. Została postawiona hipoteza mówiąca o ujemnej korelacji pomiędzy pozytywnym postrzeganiem siebie a dokonywaniem modyfikacji publikowanych zdjęć. Spodziewaliśmy się zatem wyników, które świadczyłyby o tym, że osoby lepiej postrzegające siebie (mające pozytywny stosunek do swojego ciała) będą miały mniejszą skłonność do modyfikowania swoich zdjęć, natomiast osoby nieposiadajądce dobrej relacji ze swoim ciałem - większą. Jednakże, uzyskane wyniki nie potwierdzają założeń postawionej w badaniu hipotezy. Może to stanowić sprzeczność z wynikami większości dotychczasowych badań szukających podobnych zależności. Okazuje się, że sposób, w jaki badany postrzega siebie nie zawsze wiąże się z modyfikowaniem lub niemodyfikowaniem przez niego zdjęć swojego wizerunku na platformie Instagram. Niskie postrzeganie własnego ciała nie musi wiązać się z zabiegami upiększającymi wizerunek na fotografii, zaś wysokie postrzeganie siebie niekoniecznie łączy się z niewykorzystywaniem zabiegów upiększających.
\paragraph{}
Platforma Instagram, która aktualnie stanowi najpopularniejszy serwis społecznościowy wśród nastolatków, pozwala na utrzymanie stałego kontaktu z rówieśnikami i jawne dzielenie się bieżącymi wydarzeniami ze swojego życia. Instagram to również dobre miejsce na wyrażanie siebie poprzez pokazywanie swojego ubioru, makijażu, zainteresowań, szerzenie swoich opinii i poglądów \citep{longobardi_follow_2020}. Niezależnie od tego, czy zdjęcia są modyfikowane czy nie, kreują one wizerunek człowieka, który w taki czy inny sposób może wyrażać siebie i budować obraz własnej osoby - obraz, który nie zawsze ma odwagę pokazać poza Internetem. Duży wpływ na poczucie zadowolenia z własnego wyglądu ma również porównywanie się z innymi. \citet{kleemans_picture_2018} wykazali, iż zetknięcie się przez młode kobiety ze zmodyfikowanymi zdjęciami rówieśników może prowadzić do sytuacji, w których postrzegają one siebie bardziej negatywnie. Natomiast fotografie, które nie są przez rówieśników w żaden sposób zmienione, nie muszą być  źródłem takowych skłonności. Być może samo otaczanie się osobami, które modyfikują swoje zdjęcia wiąże się z presją, aby samemu je modyfikować, żeby nie odstawać od reszty. Natomiast obracanie się w środowisku, które nie ma w zwyczaju przerabiać zdjęć swojego wizerunku może łączyć się z tym, że sami nie odczuwamy takiej potrzeby niezależnie od tego, jak prezentuje się nasze postrzeganie samego siebie (pozytywnie czy negatywnie). W związku z tym, niewykluczone, że wpływ społeczny może tu odgrywać większą rolę niż sposób, w jaki postrzegamy siebie i swoje ciało, i czy uważamy, że wymaga ono upiększenia czy nie. Z badań przeprowadzonych przez \citet{kleemans_picture_2018} i \citet{lyu_travel_2016} wynika również, że modyfikowanie swoich zdjęć może funkcjonować jako mechanizm radzenia sobie przez niektórych z negatywną samooceną. Wyniki naszego badania pokazują, że nie wszystkie młode kobiety praktykują tę metodę. Możliwe jest, że jest istnieje inny możliwy do wypracowania mechanizm radzenia sobie z tym problemem. Jest to temat wymagający przeprowadzenia dalszych badań. Ponadto na tworzenie fałszywego wizerunku w mediach społecznościowych wpływa również internalizacja ideału szczupłej  figury, korelująca z negatywną oceną własnego ciała \citep{blowers_relationship_2003}. Jednak, najnowsze badania pokazują odstępstwo od dotychczasowego kanonu piękna, głównie przez ruch \textit{body positivity} \citep{LAZUKA202085}. Dzięki temu obserwuje się przejawy zwiększonej akceptacji swojego naturalnego wyglądu, aniżeli dążenia do często nieosiągalnego ideału.
\subsection*{\normalsize{\textbf{Ewaluacja}}}
\paragraph{}
Powyższe badanie jednakże posiada swoje ograniczenia. Najważniejszym punktem, który należy poruszyć w ewaluacji badania jest fakt odrzucenia grupy kontrolnej. Nie było możliwe porównanie badanych grup z grupą kontrolną ze względu na jej zbyt małą liczebność. Z tego powodu nie jest możliwe ustalenie dokładnych zależności - istnieje prawdopodobieństwo, iż to właśnie zupełny brak publikacji zdjęć jest związany z negatywną oceną swojego ciała. Dodatkowo dobór próby został przeprowadzony metodą nieprobabilistyczną, bez użycia operatu losowania, co mogło doprowadzić do błędu próby. Ankietę otrzymały specyficzne osoby, czyli najczęściej uczniowie lub studenci z większych miast. Dodatkowo z histogramów wynika, że badanie wypełniło najwięcej osób o dość wysokiej samoocenie, co mogło wpłynąć na wyniki porównania grup modyfikującej i niemodyfikującej zdjęcia. Poza tym zebrane liczebności grup nie były zbyt porównywalne (51 do 25 osób). Ważnym punktem ograniczającym porównanie wyników z dotychczasową literaturą jest fakt przeprowadzenia badania na osobach z innej kultury niż badania wykorzystane w sformuowaniu teorii. Oprócz tego porównanie wyników ograniczają zmienne indywidualne. Brak umiejętności lub odpowiednich narzędzi do modyfikowania zdjęć też może mieć związek z częstością aplikowania takich zabiegów na swoich fotografiach.
\paragraph{}
Warto również wspomnieć o silnych stronach powyższego artykułu. W badaniach użyty został kwestionariusz nieznajdujący się w domenie publicznej i dotyczący w pełni zadowolenia z ciała, nieuwzględniając aspektów ogólnej samooceny. Dodatkowo, ankieta została stworzona tak, aby odpowiednio wybrać grupę docelową, dzięki czemu nie pojawiły się osoby, których nie dotyczyło badanie np. mężczyźni.
\newpage
\bibliography{BB}
\end{document}