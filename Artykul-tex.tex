\documentclass[12pt,a4paper,final,oneside,onecolumn,titlepage]{article}
\usepackage{times}
\usepackage{geometry}
\usepackage{setspace}
\usepackage{natbib}
\usepackage{polski}
\usepackage[utf8]{inputenc}
\usepackage[T1]{fontenc}
\usepackage[pagewise]{lineno}
\newgeometry{tmargin=2.5cm, bmargin=2.5cm, lmargin=2.5cm, rmargin=2.5cm}
\setlength{\parindent}{10mm}
\setlength{\parskip}{0mm}
\linenumbers
\doublespacing

\begin{document}
\bibliographystyle{apalike}
\begin{titlepage}
  \thispagestyle{empty}
  \begin{center}
  \vspace*{1cm}
  \Large
  \textbf{\textsc{Rola platformy Instagram w odbiorze własnego wyglądu:\\ Badanie zależności między modyfikowaniem swojego wizerunku, a oceną postrzegania swojego ciała wśród kobiet\\}}
  \vspace{1.5cm}
  \textit{Sara Pasturczak, Aleksandra Popowska, Wiktor Warchałowski\\}
  Wydział Nauk o Zdrowiu, Gdański Uniwersytet Medyczny\\
  \vspace{3cm}
  Praca zaliczeniowa z przedmiotu Metodologia Badań Psychologicznych napisana pod kierunkiem dr Małgorzaty Basińskiej\\
  \vspace{3cm}
  Gdańsk, 24 Maja 2022
  \end{center}
\end{titlepage}
\begin{center}
  \vspace*{0.5cm}
  \textbf{\textsc{Abstrakt\\}}
\end{center}
Celem niniejszego artykułu jest zanalizowanie i zbadanie problematyki związanej z wpływem reprezentacji własnego ,,ja'' w mediach społecznościowych na postrzeganie i akceptację własnego ciała u młodych kobiet. Badanie zostało przeprowadzone na 76 losowo dobranych użytkowniczkach platformy Instagram, najpopularniejszego serwisu społecznościowego wśród osób młodych. Reprezentacja własnego ,,ja'' została opisana binarnie jako modyfikowanie lub niemodyfikowanie własnego wizerunku, zaś postrzeganie własnego ciała jako wynik polskiej adaptacji 10 itemowego kwestionariusza Body Appreciation Scale 2.\\
\textit{Słowa kluczowe: media społecznościowe, postrzeganie siebie, ocena ciała}
\newpage
W dzisiejszych czasach coraz więcej osób ma dostęp do internetu, a co za tym idzie - również do portali społecznościowych, które cieszą się ogromnym zainteresowaniem. Najpopularniejszym portalem wśród amerykańskich nastolatków jest Instagram. Pozwala on na kontakty z rówieśnikami, wymianę informacji, a także wyrażanie siebie \citep{longobardi_follow_2020}. Jednakże tak powszechne użycie mediów społecznościowych prowokuje u młodych osób zachowania koncentrujące się na dążeniu do statusu w sieci (ang. \textit{digital status seeking}), czyli dążenia do popularności w Internecie mierzonej ilością polubień, komentarzy, udostępnień i obserwatorów \citep{nesi_search_2019}. Takie działania często prowadzą do zachowań ryzykownych, a samo wystawienie się na tego typu sieciową rywalizację, pozwala badaczom wysunąć wniosek, że Instagram jest najbardziej szodliwym serwisem społecznościowym. Jego nadmierne używanie może prowadzić nawet do problemów psychicznych czy zaburzeń postrzegania swojego ciała \citep{royal_society_for_public_health_status_2017}. Oprócz samego udziału młodych osób w kształtowaniu treści mediów społecznościowych, na tworzenie ich zawartości mają również wpływ czynniki socjokulturowe \citep{giorgianni_consumer_2020}. Kultura zachodnia od wieków ma tendencję do uprzedmiatawiania kobiet, a także cały czas zwraca uwagę na ważność wyglądu zewnętrznego \citep{lyu_travel_2016}. Ze względu na tak utarte kulturowe stereotypy, większość środków masowego przekazu powiela je i tworzy wizerunek ideału szczupłego ciała. Promowanie idelanych modelek i nagradzanie tego typu wizerunku w mediach jest czymś powszechnym. Stanowi to często główny powód niezadowolenia młodych kobiet ze swojego ciała, co z kolei może prowadzić do zaburzeń odżywiania \citep{grabe_role_2008}. Problemy te wynikają z dążenia do perfekcyjnego wyglądu, w celu sprostania oczekiwaniom społeczeństwa. Aby stworzyć złudzenie idealnego ciała w mediach społecznościowych, kobiety modyfikują swoje zdjęcia i często stanowi to dla nich mechanizm radzenia sobie z brakiem akceptacji siebie \citep{lyu_travel_2016}. Duży wpływ na poczucie zadowolenia z własnego wyglądu ma również porównywanie się z innymi osobami. Zetknięcie się przez młode dziewczęta ze zmodyfikowanymi zdjęciami rówieśników prowadzi do sytuacji, w których postrzegają one siebie bardziej negatywnie. Natomiast zdjęcia, które nie są przez rówieśników w żaden sposób zmienione, nie są źródłem takich skłonności \citep{kleemans_picture_2018}. Pomimo wielu przeprowadzonych badań potwierdzających szkodliwość platformy Instagram istnieją i takie, które zaprzeczają jego krzywdzącemu wpływowi. \citet{mclean_selfies_2015} w swoim badaniu pokazali, że publikowanie swoich zdjęć może wpływać na młode kobiety dwojako - mogą one zawyżać ocenę na temat swojego ciała, lub przyczyniać się do niezadowolenia z własnego wyglądu. Wyniki te mogą zaprzeczać jakiejkolwiek zależności między oceną własnego ciała a korzystaniem z mediów społecznościowych. Z tego powodu, celem niniejszego artykułu jest sprawdzenie czy fakt modyfikacji zdjęć własnego wizerunku na platformie Instagram ma związek z poziomem zadowolenia z własnego ciała u młodych kobiet. Przyjęta została hipoteza, iż fakt modyfikacji zdjęć będzie występował częściej u osób posiadających niskie zadowolenie z własnego wyglądu. Wynika to z faktu, że modyfikacja zdjęć może być mechanizmem radzenia sobie przez niektórych z negatywną samooceną \citep{kleemans_picture_2018,lyu_travel_2016}. Oprócz tego na tworzenie fałszywego wizerunku w mediach społecznościowych wpływa internalizacja ideału szczupłej sylwetki, która koreluje z negatywną oceną własnego ciała \citep{blowers_relationship_2003}.
\newpage
\bibliography{BB}
\end{document}