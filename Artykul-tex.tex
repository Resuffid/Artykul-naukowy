\documentclass[12pt,a4paper,final,oneside,onecolumn,titlepage]{article}
\usepackage{times}
\usepackage{geometry}
\usepackage{setspace}
\usepackage{fancyhdr}
\usepackage{natbib}
\usepackage{polski}
\usepackage[utf8]{inputenc}
\usepackage[T1]{fontenc}
\usepackage[pagewise]{lineno}
\newgeometry{tmargin=2.5cm, bmargin=2.5cm, lmargin=2.5cm, rmargin=2.5cm}
\setlength{\parindent}{10mm}
\setlength{\parskip}{0mm}
\linenumbers
\doublespacing

\begin{document}
\pagestyle{fancy}
\chead{\textsc{Rola platformy Instagram w odbiorze własnego wyglądu}}
\lhead{\thepage}
\bibliographystyle{newapa}
\begin{titlepage}
  \thispagestyle{empty}
  \begin{center}
  \vspace*{1cm}
  \Large
  \textbf{\textsc{Rola platformy Instagram w odbiorze własnego wyglądu:\\ Badanie zależności między modyfikowaniem swojego wizerunku, a oceną postrzegania swojego ciała wśród kobiet\\}}
  \vspace{1.5cm}
  \textit{Sara Pasturczak, Aleksandra Popowska, Wiktor Warchałowski\\}
  Wydział Nauk o Zdrowiu, Gdański Uniwersytet Medyczny\\
  \vspace{3cm}
  Praca zaliczeniowa z przedmiotu Metodologia Badań Psychologicznych napisana pod kierunkiem dr Małgorzaty Basińskiej\\
  \vspace{3cm}
  Gdańsk, 24 Maja 2022
  \end{center}
\end{titlepage}
\end{document}